\section{Analisi dei Requisiti}
	Si vuole realizzare una base di dati che organizzi le inserzioni di vendita in un sito web. \`{E} quindi necessario definire, prima di tutto, l'oggetto protagonista di questo DB: l'entit\`{a} \textbf{moto}. 		\subsection{Moto}	
		La raccolta dei requisiti per definire l'entit\`{a} \textbf{moto} ci ha permesso di individuare le seguenti caratteristiche:
	\openup 0em
	\begin{itemize}
	\setlength\itemsep{0em}
		\item il nome della marca e del modello;
		\item l'anno di produzione e la data d'immatricolazione;
 		\item il colore e la categoria a cui appartiene la moto (enduro, cross, da corsa, custom, ...);
  		\item la classe di emissioni, la cilindrata e la potenza erogata dal motore (in cavalli);
  		\item il kilometraggio ed il numero totale di proprietari;
  		\item la data dell'ultimo tagliando e della prossima revisione da effettuare.;
	\end{itemize}
  		