\section{Analisi dei Requisiti}
	Si vuole realizzare una base di dati che organizzi le inserzioni di vendita in un sito web. \`{E} quindi necessario identificare, prima di tutto, le entit\`{a} protagoniste di questo DB.
	
	\subsection{Moto}	
		Una moto \`{e} l'oggetto principale di ogni \textbf{inserzione}, la quale \`{e} creata da un \textbf{utente}. La raccolta dei requisiti per definire l'entit\`{a} \textbf{moto} ci ha permesso di individuare le seguenti caratteristiche:
	\begin{itemize}
	\openup -0.5em
		\item il nome della marca e del modello;
		\item l'anno di produzione e la data d'immatricolazione;
 		\item il colore e la categoria a cui appartiene la moto (enduro, cross, da corsa, custom, ...);
  		\item la classe di emissioni, la cilindrata e la potenza erogata dal motore (in cavalli);
  		\item il kilometraggio ed il numero totale di proprietari;
  		\item la data dell'ultimo tagliando e della prossima revisione da effettuare.
	\end{itemize}
	
  	\subsection{Utente}
  	Un utente \`{e} colui che inserisce l'annuncio della moto che vuole vendere nel sito web (un utente deve poter inserire pi\`{u} annunci, inerenti per\`{o} a moto diverse. Sono stati, quindi, individuate le seguenti propriet\`{a}:
  	\begin{itemize}
  	\openup -0.5em
  		\item le sue generalit\`{a}, come il nome ed il cognome; 
  		\item un recapito telefonico, mail, regione e provincia di residenza;
  		\item il fatto che sia admin del sito oppure no.
  	\end{itemize}
  	
  	\subsection{Regione}
  	
  	
  		